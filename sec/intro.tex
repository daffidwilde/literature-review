\section{Introduction}\label{sec:intro}

Modern research, in all respects, is highly dependent on the use of data. The
authors of \cite{Womack2015} consider a stratified sample of several thousand
articles from leading journals across the natural sciences, and suggests that
perhaps as many as 80\% of all scientific articles published in 2014 utilised
research data directly. That article is one of many in an increasing body of
publications over the last decade that concern themselves with the state of
research data~\cite{Higman2019}, its utilisation and
openness~\cite{Aslam2017,Zuiderwijk2020}, and the importance of best
practices~\cite{Colavizza2020,Corti2019}. These publications focus on using this
set of characteristics as a measure of the quality of a research data source,
which also acts as a mark of the associated research's quality; this is in
contrast to other recent trends where the pursuit of more voluminous and varied
data sources has taken precedence~\cite{Batistic2019}. In~\cite{Stall2019}, the
authors argue for the widespread use of a concept (used already in the
geosciences) that unifies these characteristics: that research data should be
\emph{Findable, Accessible, Interoperable and Reusable} (FAIR) --- an acronym
coined in~\cite{Wilkinson2016}.

In tandem with this shift was the rise in the use of software to implement and
compute algorithms, and so electronic data became essential. Similarly to the
FAIR guidelines for research data, best practices for research software
development have been adopted~\cite{Aberdour2007,Benureau2018,Jimenez2017}. As
the size and complexity of research data increased, so did the span of the field
\emph{machine learning}. Machine learning is a term to describe how a computer
(machine) may learn (by use of statistics) from a potentially large source of
data, without following explicit instructions. Owing in part to this broad
definition, machine learning comprises a great deal of techniques that were
borne out of statistics, including regression, classification and clustering.

Healthcare modelling is one crucial (albeit broad) branch of research in which
decisions are increasingly being informed by data-driven
methodologies~\cite{Alexander2018,Belle2015,RiosZertuche2020}, often based
around machine learning techniques. Clustering is a long-standing and often-used
% TODO Add a catch all reference for clustering?
method in healthcare settings; its efficacy --- as a tool not only to expose
homogeneous partitions within a dataset, but to straightforwardly garner the
attention of non-technical stakeholders --- has proven it to be an essential
part of healthcare modelling research, as is discussed in
Section~\ref{subsec:healthcare}.

Regardless of the particular methods that are to be implemented, the quality of
a methodology must be evaluated as being fit for purpose before it can be used
in a real world setting. The ways in which this evaluation is carried out is
reliant on both the methods themselves and the setting in which they are
applied, but there are some approaches used across research fields such as
consensus by literature or beating some rival method(s) according to a metric on
a dataset.

This survey considers the literature surrounding these three components of
modern operational and mathematical research --- clustering, the evaluation of
algorithms, and healthcare modelling --- as well as their intersections. Given
the potentially vast nature of the literature under study, formulating a
reasonable slice that encompasses the state of the research is difficult. As
such, this survey utilises two approaches, setting the following structure:

\begin{itemize}
    \item In Section~\ref{sec:review}, a review of a selection of representative
        and innovative publications from across the aforementioned research
        topics;
    \item In Section~\ref{sec:bibliometric}, a software-based bibliometric study
        of the literature from five academic publishers and preprint servers.
\end{itemize}
