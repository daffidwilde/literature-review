\section{Literature review}\label{sec:review}

This section comprises what could be considered a classical literature review:
an analysis of selected articles and publications that constitute the state and
progress of each research topic.

% TODO Would a diagram of some sort be nice here?

\subsection{Clustering}\label{subsec:clustering}

Clustering, or \emph{cluster analysis}, is a generic term used to describe a
number of techniques whose objective is to partition some dataset into parts
(clusters) according to some distance (similarity) measure. The generated
partition should be such that the members of one cluster are more similar to one
another than the rest of the data~\cite{Everitt2011}. This form of analysis has
found applications in a multitude of fields, including recently the optimisation
of energy systems~\cite{Jing2019,Teichgraeber2019}, wireless sensor
networking~\cite{Goswami2019}, and the biological
sciences~\cite{Bulut2020,Kiselev2019}.

Clustering originated in the social sciences around the turn of the
20\textsuperscript{th} century in a similar manner to several other (now
ubiquitous) statistical tools, including hypothesis testing and \(p\)-values,
correlation, and factor analysis. Early work on cluster analysis, such
as~\cite{Cattell1943,Driver1932}, focused on its applications in attempting to
identify traits which led to cultural and psychological differences between
groups of people, as opposed to studying the methods used to identify any
partitions.

The focus of clustering research shifted to the statistical nature of the
methods following work on clustering under the name \emph{numerical
taxonomy}~\cite{Sneath1973,Sokal1966}. These works were highly influential and
led to an abundance of research into clustering methods,
including~\cite{Diday1976} and~\cite{Hartigan1975} which each formalised the
principles of clustering as a part of statistics. Furthermore, this period
witnessed the advent of seminal work on clustering algorithms such as
\(k\)-means~\cite{Hartigan1979} and hierarchical
clustering~\cite{Defays1977,Sibson1973} that are still considered fundamental
methods.
% TODO Can we have a reference or 2 here to backup claim that they're
% still fundamental? (Perhaps even the initialisation paper of yours.
Since then, with an increase in the data under study, clustering has fallen more
squarely into the category of machine learning techniques that perform
\emph{unsupervised learning} --- this is because clustering relies solely on the
entirety of the observed data and some a priori knowledge such as a set of
parameters~\cite{Dayan1999}.

The remainder of this section offers a summary of the three principle types of
clustering algorithms: those belonging to the \(k\)-means (centroid-based)
paradigm, hierarchical models, and, finally, density-based algorithms. In
addition to these, fuzzy clustering and model-based clustering are popular. For
overviews and reviews of the methods belonging to the former,
consider~\cite{Ferraro2019,Gosain2016,Li2016},
and~\cite{Bouveyron2019,Fruhwirth2019,McNicholas2016} for the latter.

\subsubsection{Centroid-based clustering}\label{subsubsec:kmeans}

The concept of \(k\)-means clustering is to partition a
real-valued dataset into \(k \in \mathbb N\) homogeneous parts (clusters), where
each data point is assigned to the cluster to which the point is closest. The
sum of these within-cluster distances is often referred to as the \emph{inertia}
of the clustering. The distance from a point to a cluster is defined as the
distance from the point to the mean of the elements in that cluster, also
referred to as the cluster centre or centroid; this alternative name gives
another name to the paradigm, \emph{centroid-based clustering}. Typically, the
distance measure used is the Euclidean distance, with the assumption that
several preprocessing techniques are used to mitigate any scaling discrepancies
in the attributes of the data.

Although \(k\)-means clustering has independently been discovered a number of
times since as early as the 1950s (comprehensive histories on the subject can be
found in~\cite{Bock2007,Jain2010}), its formulation is commonly attributed
to~\cite{Hartigan1979}. Given its lengthy standing, there are a multitude of
algorithms that perform \(k\)-means clustering. However, the most commonly used
is Lloyd's algorithm~\cite{Lloyd1982}. This procedure is remarkably
uncomplicated in its statement, and has now become so synonymous with
\(k\)-means clustering that it is referred to as `the \(k\)-means algorithm'. To
summarise it in a sentence, the algorithm iteratively partitions the numerical
data into \(k\) parts, reassigning data points and adjusting cluster means until
no point moves cluster on a full cycle of the dataset. The algorithm has proven
popular due to this simplicity. Furthermore, it is scalable and can be
parallelised~\cite{Bahmani2012}, and its implementations are
concise~\cite{Olafsson2008,Wu2009}.

However, centroid-based clustering is not without its drawbacks. The most
notable of these is its dependency on being presented with data that can be
naturally partitioned into \(k\) clusters of a particular
type~\cite{Ostrovsky2013}. Namely, the true clusters should be isotropic, convex
and linearly separable. The reason these conditions exist, particularly for
\(k\)-means, is because the \(k\)-means (Lloyd's) algorithm is a heuristic for
identifying the centroidal Voronoi tessellation of a dataset~\cite{Du2006}.
Hence, the true objective of the algorithm is to divide the attribute space into
\(k\) equally weighted parts using the dataset as a sample, rather than focusing
on dividing the dataset itself.

Regardless of the true nature of a centroid-based clustering algorithm, there is
the issue of choosing a suitable value of \(k\) prior to clustering. A common
approach to this problem is the so-called `elbow' method where a clustering
algorithm is run using a range of values for \(k\)~\cite{Aggarwal2013}. This
range is typically informed by what is considered reasonable in the problem
domain. From these runs, the value of \(k\) is plotted against some objective
and the plot is inspected for a kink or `elbow'. Typically, inertia is used, but
another popular metric is the silhouette coefficient: a measure for both the
intra-cluster tightness and inter-cluster separation of a
clustering~\cite{Rousseeuw1987}.

Irrespective of the objective used, the definition of what constitutes an elbow
is not necessarily well-defined. The works \cite{Sujatha2013,Syakur2018} each
employ the elbow method with final inertia as the objective. In each case, the
authors offer little explanation of their methodologies other than choosing
\(k\) where `drastic' changes occur in the objective. Attempts to formalise this
method have been made, however. For instance, the authors of \cite{Satopaa2011}
provide a `knee-point' detection algorithm that finds the point of maximal
curvature in the domain of a function when provided with some basic properties
of that function (whether it is convex, monotonic, etc.). This algorithm has
been adapted to cleanly carry out the elbow method in software packages, such as
Yellowbrick~\cite{Bengfort2019}, by interpolating the objective values.

Further to this formalisation of the elbow method, other extensions to
\(k\)-means itself have been presented to alleviate all dependency on making a
choice for \(k\). For example, in~\cite{Olukanmi2019}, the authors propose an
automatic decision scheme for \(k\)-means based on a vector comprised of the
values that constitute the inertia of a clustering. The process determines that
if this vector meets some criterion attached to its standard deviation, then
each cluster is sufficiently unimodal and Gaussian, fulfilling the condition
that clusters be roughly spherical.

Another limitation of \(k\)-means is that it is dependent on its initial
solution. The standard initialisation process is to select \(k\) points in the
dataset at random, introducing a stochastic element to the algorithm. Studying
methods to find sensible (ideally, deterministic) initial solutions has formed a
substantial part of centroid-based clustering research. For instance, the
authors of~\cite{Manoharan2016} provide a novel initialisation that pursues the
Voronoi tessellation by use of a `divide-and-conquer' approach. Under the
presented scheme, the initial centroids are selected based on a rough partition
of the attribute space according to their extreme values.  Likewise, the
initialisation presented in \cite{Singh2013} assigns each centroid using the
arithmetic mean and dividing the remaining attribute space in two.  Meanwhile,
the method introduced in~\cite{Katara2015} initialises \(k\)-means by sorting
the instances in the dataset according to their distance from the origin. Once
sorted, the instances are split into \(k\) disjoint sets and the mean of each
set is assigned as a centroid. A likely issue with this approach is that there
are new assumptions about the data, i.e. that it is centred about the origin in
all dimensions (rectifiable through feature scaling), and that all points
equidistant from the origin are similar to one another.

A crucial part of centroid-based clustering research is dedicated to alternative
measurements for the centre of a cluster. A cluster centroid is broadly regarded
as a purely geometric object, and is measured using a distance metric and a
measure of central tendency. Using the Euclidean mean as the centre of a cluster
is a sensible choice when clustering continuous, normalised data. However, if
the data is not scalable (if it is categorical, for instance) then this approach
is not suitable. Further, the Euclidean distance is linear, leading \(k\)-means
to be prone to the effects of outliers in the data. As such, numerous
extensions to Lloyd's algorithm exist in the literature that are designed to be
more robust to these limitations. For continuous (or at least ordered) data,
\(k\)-medians and \(k\)-medoids exist. The former uses the median value of each
attribute in a cluster to form the centre~\cite{Arya2001,Bradley1997},
mitigating the effect of outliers.

In \(k\)-medoids, the centre of a cluster is taken to be the data point
which is closest to all other points in the cluster~\cite{Kaufman1987}.
Typically, \(k\)-medoids makes use of the Manhattan distance, but is generic in
its schema. If Euclidean distance is used, then \(k\)-medoids is identical to
\(k\)-means, with the restriction that the centre of the cluster is an actual
point in that cluster. Specifying that the centre be a real point is seen as the
primary benefit of \(k\)-medoids, and continues to be actively
studied~\cite{Schubert2019,Ushakov2021}. There are cases where a virtual cluster
centre is not sufficient, such as facility allocation~\cite{Chen2016,Wang2020}
and the clustering of gene sequences~\cite{Johnson2018}.

When considering a categorical or mixed-type dataset, none of the aforementioned
clustering techniques will work as they were intended because the sense of order
does not exist across the entire attribute space. While it is possible to use
\(k\)-means in these situations, by converting any categorical attributes into
pseudo-numerical (binary) attributes by use of dummy coding, this is not
recommended. One potential failing is the inflated effect of categorical
variables with many distinct values in any distance calculation, as these will
produce more binary variables. Another is that by introducing potentially many
binary variables, the dimensionality of the dataset increases substantially, and
so the points in the higher-dimension form of the data become intrinsically
further from one another, thus loosening any sense of cluster homogeneity.

Therefore, another approach should be taken that handles the data directly. The
\(k\)-modes and \(k\)-prototypes algorithms (introduced in~\cite{Huang1998}) are
capable of handling categorical and mixed-type data, respectively. The
\(k\)-modes algorithm is an extension to Lloyd's algorithm, like \(k\)-medians
and \(k\)-medoids, where the cluster centre is defined as a point whose
attribute values are most frequent among those of the points in a
cluster.~\cite{Huang1998} provides a so-called `matching dissimilarity' measure
to expedite the centre calculations, making \(k\)-modes scalable and
computationally efficient~\cite{Madhuri2014}. The \(k\)-prototypes algorithm is
essentially a blended form of \(k\)-modes and \(k\)-means where each method is
applied to the categorical and numeric attributes at each iteration, before
their respective costs are combined linearly according to some parameter,
\(\gamma\). Work into \(k\)-modes clustering is largely focused on the initial
choice of centroids, as
in~\cite{Cao2009,Jiang2016,Khan2013,Khan2007,Taoying2013,Wilde2020}, and
improving on the original distance measure. Two popular dissimilarity measures
for \(k\)-modes clustering were presented in~\cite{Cao2012} and~\cite{Ng2007}.
Each of these novel measures improve on the basic metric by taking into account
the relative frequency of the values of each attribute. The latter
measure~\cite{Ng2007} focuses locally to the cluster at hand, while the
former~\cite{Cao2012} considers their frequencies from a universal perspective.

While centroid-based clustering is dominated by Lloyd-like algorithms, recent
work has sought out alternative approached. For instance, the authors
of~\cite{Chen2019} revise centroid clustering from an expectation-maximisation
process to one that considers its data points as agents that act with
preferences based on their locale. The overall aim of this approach is to
incorporate a sense of mathematical fairness by identifying a solution that can
not be justifiably rejected by any subset of its agents. Such a solution is
found through a proportional (i.e. Pareto-optimal) distribution of the cluster
centres according to some distance metric.

At this point, the survey returns to the effect of outliers on Lloyd-like
algorithms, especially \(k\)-means, and the literature around reducing that
effect. The \(k\)-means-sharp algorithm was presented in~\cite{Olukanmi2017};
this revision to \(k\)-means embeds an outlier detection feature which relies
solely on the underlying structures from the original \(k\)-means algorithm. The
benefit of handling adverse noise in this way is twofold: first, it preserves
the computational efficiencies that come with using the mean (as opposed to
sorting the attributes of each cluster to find a median in \(k\)-medians, say);
and, second, there are no additional parameters to determine.
In~\cite{Gupta2017}, the authors provide another integration with \(k\)-means
for detecting outliers by way of a local search, but in order to do so, they
introduce another parameter determining the cardinality of the neighbourhood of
a point. In contract to these extensions, are examples of works that exploit
this weakness of \(k\)-means clustering~\cite{Lei2012,Wei2019}. In each case,
the \(k\)-means algorithm partitions a dataset, and outliers are identified
according to some neighbourhood threshold associated with the elements of a
cluster.

The final remark in this section of the survey is concerned with the relaxation
of the constraints on cluster geometry by centroid-based clustering algorithms.
This particularity is fundamental to how these algorithms operate and are
designed, and so little research has been done on the subject. Having said that,
the work in~\cite{Sung1998} demonstrates how the Mahalanobis distance can be
used in place of the Euclidean distance to circumvent the dependency on
spherical (i.e.  isotropic) clusters by \(k\)-means. However, the resultant
clusters must still be convex and linearly separable. The Mahalanobis distance
is commonly used to detect outliers in other settings, and is dependent on the
eigenvalues of the covariance matrix derived from a
dataset~\cite{Mahalanobis1936}. A potential downside to this measure is the
singularity of that matrix, which is particularly likely in high-dimensional
data. Another exemplar work is~\cite{Statman2020}, where a variant of
\(k\)-means is presented as being agnostic to its distance measure, and removes
the requirement for the attribute space to be a metric space at all.

\subsubsection{Hierarchical clustering}

Hierarchical clustering seeks to identify a hierarchy of the relationships
between subsets of points in a dataset. A hierarchical clustering algorithm
depends on a distance metric for measuring inter-point similarity, and a
\emph{linkage criterion} to determine the distance between two sets of points.
For the most part, algorithms for hierarchical clustering are one of two types:
agglomerative or divisive~\cite{Kaufman1990}. Agglomerative algorithms,
originating with~\cite{Johnson1967,Ward1963}, begin with each point in its own
cluster and work to merge clusters together, constructing the hierarchy from the
bottom up. Meanwhile, divisive algorithms, with~\cite{Edwards1965} being an
early example, take a top-down approach and seek to split the dataset into
reasonably homogeneous clusters.

Since its origins, hierarchical clustering research has been dominated by
agglomerative methods; this is due, largely, to the computational issues
associated with how to best split a heterogeneous dataset into parts
recursively. For a dataset of \(n\) points, a complete search of the possible
splits takes \(2^n\) calculations, and so divisive methods require efficient
heuristics to be even remotely plausible on real-world data. A common choice is
the \(k\)-means algorithm~\cite{Moseley2017,Peterson2018} which was discussed in
Section~\ref{subsubsec:kmeans}.

In addition, hierarchical clustering has been successfully reframed as a
combinatorial optimisation problem, as is done with divisive clustering
in~\cite{Dasgupta2016}, which allows for the application of optimisation
heuristics such as linear programming~\cite{Roy2017}, local
search~\cite{Aljarah2019} and probabilistic estimation~\cite{Fan2015} to
identify a clustering in reasonable time. Furthermore, related works exist which
concern themselves with determining the quality of a hierarchical structure with
respect to some cost function, like~\cite{Bilu2012,Lyzinski2017}.
In~\cite{CohenAddad2018}, the authors combine these methodologies to produce a
divisive clustering algorithm that achieves logarithmic complexity.

While heuristic methods exist for agglomerative
clustering~\cite{Aljarah2019,Fan2015}, its methods do not bear the same burdens
as divisive methods; the grouping together of parts can be achieved with more
straightforward tools. Generally, the definition of an agglomerative
algorithm is reliant on their linkage criterion. Meanwhile, the
distance metric used is largely dependent on the type of data
presented~\cite{Nielsen2016}. Two of the fundamental criteria for hierarchical
clustering are complete linkage, which defines the CLINK
algorithm~\cite{Defays1977}, and single linkage, providing the SLINK
algorithm~\cite{Sibson1973}. For two clusters, \(X\) and \(Y\), and a point-wise
distance metric, \(d\), each criterion, \(D\), is evaluated as follows:

\begin{itemize}
    \item Complete linkage: \(D(X, Y) = \max_{x \in X, y \in Y} d(x, y)\)
    \item Single linkage: \(D(X, Y) = \min_{x \in X, y \in Y} d(x, y)\)
\end{itemize}

These definitions give each approach the colloquial names \emph{farthest
neighbour} and \emph{nearest neighbour} clustering, respectively. Moreover,
defining linkage in this way allows a hierarchical (agglomerative or divisive)
algorithm to be run using only a point-wise distance matrix, as opposed to a
dataset directly, allowing for increases in computational efficiency by use of
caching~\cite{Nielsen2016}. However, hierarchical methods remain prone to
outliers and are biased towards globular clusters because of definitions such as
these. Ongoing reviews on the subject of hierarchical (although predominantly
agglomerative) clustering methods are
available~\cite{Murtagh1983,Murtagh2012,Murtagh2017}.

Like centroid-based clustering, a principle issue with hierarchical clustering
is knowing how many clusters is the most appropriate number. Hierarchical
algorithms terminate when all points exist in a single cluster (for
agglomerative methods) or are all separate (for divisive). A key advantage of
clustering in this way is the retention of a complete history of how the points
relate to one another. These histories are used to visualise the clustering
process via a dendogram. A dendogram is an acyclic graph (tree) arranged into
levels corresponding to the stages at which two clusters are merged (or split).

With such a visualisation, and the underlying tree itself, an appropriate number
of clusters can be identified after running the algorithm once, as opposed to
the ranges of runs required for the elbow method in centroid-based clustering.
The authors of~\cite{Tellaroli2016} present a hierarchical clustering schema
that incorporates complete-linkage clustering with the minimum variance
criterion presented in~\cite{Ward1963}. In doing so, the method automates the
process of choosing an appropriate number of clusters; a decision that is
usually made post hoc according to another metric, such as the silhouette
coefficient, evaluated at each level of the tree.
% TODO I wonder if a figure  with a tree would be nice here (on an actual data
% set). And likewise we could have one of those classic k-means scatter plots in
% the previous section?

\subsubsection{Density-based clustering}

Density-based clustering differs from centroid-based methods in that
it actively seeks more of the underlying structure of a dataset. In
centroid-based clustering, decisions are made using the distance between two
points, which provides a \emph{flat} clustering without an underlying structure.
Meanwhile, hierarchical methods make use of identification and aggregation
devices, which are derived from point-wise distances, to measure and organise
the connectivity of subsets in a dataset, providing insight into the structure
of the data. Arguably, density-based clustering takes this structural approach a
step further and considers the attribute space of a dataset as a density
surface. Then, a cluster is defined as a high-density region of that surface.

By considering a dataset in this way, the clusters can be of arbitrary shape,
overcoming the major shortfall of the other paradigms considered
here~\cite{Raykov2016}. Furthermore, recognising data points in sparse areas of
the surface (i.e. with low density) often affords density-based methods the
automatic filtration of noise, without having to implement a specific schema. As
such, several methods splicing density-based clustering with the other paradigms
have been offered. For example, (Fast)DPeak~\cite{Chen2020},
HDBSCAN~\cite{Campello2013} and DenPEHC~\cite{Xu2016} incorporate hierarchical
structures with density-based decision making. Likewise,
KIDBSCAN~\cite{Tsai2006} utilises \(k\)-means clustering to identify centres
with a high density.

Density-based clustering is relatively young when compared to other paradigms,
only emerging at the end of the 20\textsuperscript{th} century with the DBSCAN
algorithm~\cite{Ester1996}. Given its youth, contemporary clustering literature
surveys (such as~\cite{Jain1999,Xu2005}) lacked thorough study of density-based
clustering methods. Over time, specific surveys have been published, culminating
in a most recent survey~\cite{Bhattacharjee2020} which provides an exceptionally
detailed review of density-based clustering algorithms. This survey includes a
descriptive taxonomy of 32 density-based methods, categorised by their density
definition, parameter sensitivity, mode of execution and the nature of the data
being clustered.

DBSCAN is a point-based, parameter-sensitive clustering algorithm that has
proven popular, and is now synonymous with density-based clustering. The initial
work has spurred a slue of extensions to DBSCAN which aim to bypass its
shortcomings, including Generalised DBSCAN~\cite{Sander1998}, Incremental
DBSCAN~\cite{Bakr2015,Ester1998}, Improved DBSCAN~\cite{Borah2004}, and various
parallelised versions of the algorithm~\cite{Bohm2009,He2011,Loh2014,Xu2002} to
overcome its time-complexity. The primary issues with DBSCAN relate to its
parameters: a radius, \(\epsilon > 0\), defining the neighbourhood of a point,
and an integer, \(\text{Minpts} \in \mathbb N\), that specifies the minimum
cardinality of a neighbourhood required for a point to be considered dense.
These parameters correspond to DBSCAN being point-based: regions of high density
are determined using the points directly. The first disadvantage of DBSCAN is in
estimating these parameters. The seminal work on DBSCAN~\cite{Ester1996}
includes some default parameters based on dimensionality, but a good
understanding of the dataset at hand is required for effective use. This issue
leads to the second, which is that the parameters are global in scope, meaning
that DBSCAN is unable to identify clusters that are of varying densities.
Finally, the performance of DBSCAN is limited in high-dimensional space, because
of the so-called \emph{curse of dimensionality}~\cite{Keogh2017} --- the same is
true of other clustering algorithms, including \(k\)-means.

However, these limitations are not without an alternative. For instance,
CLIQUE~\cite{Agrawal1998} and OPTIGRID~\cite{Hinneburg1999} each provide
density-based clustering algorithms that are designed to perform well in
high-dimensional space. Each of these algorithms is grid-based (as opposed to
point-based), so the attribute space is discretised into rectangles. These
rectangles are then used to identify dense regions. DCore~\cite{Chen2018}
provides powerful point-based clustering in arbitrarily high dimensions.

To identify clusters of variable densities, DVBSCAN~\cite{Ram2010} and
HDBSCAN~\cite{Campello2013} are available. The latter aggregates outputs from
DBSCAN using a range of radii, making it more resilient to changes in either
parameter. The former runs DBSCAN with the addition of allowing a cluster to
expand provided its core points' local (\(\epsilon\)-neighbourhood) density
satisfy some criterion; because of this, it requires careful consideration of
its parameters to perform well.

Lastly, any parameter-adaptive, density-based algorithm will do away with the
trouble of choosing precise parameters. Examples include the popular OPTICS
algorithm~\cite{Ankerst1999}, DBCLASD~\cite{Xu1998} for spatial data, and
DSets-DBSCAN~\cite{Hou2016}. The last provides a clustering that is independent
of the parameters provided, but is exclusively for the clustering of images.
% TODO Some sort of figure to complement the other two?


\subsection{Healthcare modelling}\label{subsec:healthcare}

Healthcare modelling is a broad term that encompasses a plethora of techniques
from a number of disciplines such as financial modelling and forecasting, and
operational problems like vehicle routing or staff rostering. This review
focuses on these operational pursuits, and particularly those that are concerned
with patients directly. The decision to narrow the literature in this way is
both practical and conscientious, but an array of comprehensive reviews are
available, including~\cite{Brailsford2016,Galetsi2020,Kunc2018,Palmer2018}.
Practical by reducing the span of literature on `healthcare modelling' to
something less cumbersome, and conscientious as it allows this review to make a
small contribution to the research of the progressively more commonplace concept
of patient-centred care.

This form of healthcare, formally defined in~\cite{Robinson2008}, demands that
the perspective of a healthcare system should align itself with its patients'
needs and lived experiences. The alternative to patient-centred care would be a
system in which patients are treated in an exact but vague way, according to
only the needs of the system, say. Some form of patient-centred care has been
adopted in healthcare systems around the world including both state-funded and
private systems~\cite{DoH2010,Dewi2013,Luxford2011}. Unsurprisingly, its
application has been commended by patients, advocates and practitioners alike
for improving condition-specific
populations~\cite{Foster2019,Gambling2010,Gondek2016,Tsianakas2012} and more
broadly~\cite{IAPO2012,Richards2015,Santana2019}.

The remainder of this section considers two aspects of healthcare modelling
research that directly contribute to the advancement of patient-centred care:
segmentation analysis and the modelling of queues. The former is often achieved
through clustering, and could be regarded as a direct application of cluster
analysis to healthcare populations. The latter is a broader form of analysis
with use cases in a variety of healthcare problems, including resource
utilisation~\cite{Prokofyeva2020}, estimating bed capacity~\cite{Williams2015},
and the taxonomy of patient pathways~\cite{Rojas2016}.


\subsubsection{Segmentation analysis}

Segmentation analysis allows for the targeted analysis of otherwise
heterogeneous healthcare datasets and encompasses several techniques from
operational research, statistics and machine learning. One of the most desirable
qualities of this kind of analysis is the ability to glean and communicate
simplified summaries of patient needs to stakeholders within a healthcare
system~\cite{ElDarzi2009,Tomar2013,Vuik2016b,Yoon2020}. For instance, clinical
profiling often forms part of the broader analysis where each segment is
summarised in a phrase or infographic~\cite{Vuik2016a,Yan2019}.

The survey identified three commonplace groups of patient characteristics used
to segment a patient population: system utilisation metrics; clinical
attributes; and the pathway. The last is not used to segment the patients
directly, but instead groups patients' movements through a healthcare system;
this is typically done using a technique known as process mining. This technique
originates in business analytics, and has been used to study the efficiency of
hospital systems~\cite{Arnolds2018,Delias2015}. The remaining characteristics
can be segmented in a variety of ways, but recent works tend to favour
unsupervised methods --- typically latent class analysis (LCA) or
clustering~\cite{Yan2018}.

LCA is a statistical, model-based method used to identify groups (called latent
classes) in data by relating its observations to some unobserved (latent),
categorical attribute. This attribute has multiple possible categories, each
corresponding to a latent class. The discovered relations enable the
observations to be separated into latent classes according to their maximum
likelihood class membership~\cite{Hagenaars2002,Lazarsfeld1968}. This method has
proved useful in the study of comorbidity patterns (as
in~\cite{Kuwornu2014,Larsen2017}) where combinations of demographic and clinical
attributes are related to various subgroups of chronic diseases.

As demonstrated in Section~\ref{subsec:clustering}, clustering includes a wide
variety of methods where the common theme is to maximise homogeneity within, and
heterogeneity between, each cluster. Of those methods, \(k\)-means clustering
(or a variant thereof) is the most widely used in
healthcare~\cite{%
    Elbattah2017,Haraty2015,Ogbuabor2018,Santhi2010,Silitonga2018,Vuik2016a%
}; this is likely due to its simplicity and scalability. Hierarchical clustering
methods have also been applied to operational healthcare problems in recognising
patient utilisation patterns~\cite{Zayas2016}, profiling their healthcare
preferences~\cite{Liu2009}, and in mapping out effective healthcare leadership
models~\cite{Hargett2017}.

Furthermore, in \cite{Vuik2016a}, hierarchical clustering is utilised as a part
of the formative analysis, identifying a suitable number of clusters.  In
addition, supervised hierarchical segmentation methods such as classification
and regression trees (as in~\cite{Harper2006,Kumar2019}) have been used where an
existing, well-defined label or attribute is of particular significance. A crucial
and attractive reason for using hierarchical methods in healthcare settings comes
from the dendogram visualisation. The use of effective visualisation tools
encourages (and allows, in some cases) involvement by key, expert stakeholders in
the simulation process. The same is true of \(k\)-means for its straightforward
construction. The authors of~\cite{Jahangirian2015} present a analysis of the
factors which result in a low level of engagement from stakeholders with
healthcare simulation work. The key findings indicate that complexity and
communication are the limiting factors for stakeholders, and so the onus resides
with researchers to make their models informative, effective and transferable.

\subsubsection{Queuing models}\label{subsubsec:queuing}

A queue models the arrival (and exit) of customers to (and from) a point of
service. A queuing network describes how two or more queues may interact with
each other. Since the seminal works by Erlang~\cite{Erlang1917,Erlang1920}
established the core concepts of queuing theory, the application of queues and
queuing networks to real services has become abundant, including healthcare
services. By applying these models to healthcare settings, many aspects of the
underlying system can be studied. A common area of study in healthcare settings
is of service capacity. The study~\cite{McClain1976} is an early example of such
work where acute bed capacity was determined using hospital occupancy data.
Meanwhile, more modern works consider more extensive sources of data to build
their queuing models~\cite{Palvannan2012,Pinto2014,Williams2015}. Moreover, the
output of a model is catered more towards being actionable --- as is the
prerogative of operational research. For instance, in~\cite{Pinto2014}, the
authors devise new categorisations for both hospital beds and arrivals that are
informed by the queuing model. A further example is~\cite{Komashie2015} where
queuing models are used to measure and understand satisfaction among patients and
staff.

In addition to these theoretic models, healthcare queuing research has expanded
to include computer simulation models. The simulation of queues, or networks
thereof, have the benefit of adeptly capturing the stochastic nuances of
hospital systems over their theoretic counterparts. Example areas include the
construction and simulation of Markov processes via process
mining~\cite{Arnolds2018,Prokofyeva2020,Rebuge2012}, and phase-type patient
flow analysis~\cite{Bhattacharjee2014,McClean2011}.

Regardless of the advantages of simulation models, a prerequisite for simulation
research is having access to reliable software with which to construct those
simulations. A common approach to building simulation models of queues is to use
a graphical user interface such as Simul8. These tools are designed to be highly
visual, making them attractive to organisations looking to implement queuing
models without necessary technical expertise, including the NHS. The authors
of~\cite{Brailsford2013} discuss the issues around operational research and
simulation being taken up in the NHS despite the availability of intuitive software
packages like Simul8.

However, they do not address a core principle of good simulation work:
reproducibility. The ability to reliably reproduce a set of results is of great
importance to scientific research but continues to be an issue in simulation
research generally~\cite{Fitzpatrick2019,Taylor2018}. When considering issues with
reproducibility in scientific computing (simulation included), the source of any
concerns is often with the software used~\cite{Ivie2018}. Using well-developed,
open-source software can alleviate issues around reproducibility and reliability
as how they are used involve less uncertainty and require more rigour than
`drag-and-drop' software. One example of such a piece of software is the discrete
event simulation library, Ciw~\cite{Palmer2019}.

The simulation of queues (or networks thereof) also makes the modelling of
complex scenarios more practical and accessible. One scenario of interest in
real-world settings is a queue with multiple customer classes. In a multi-class
queue, each class of customer has its own set of attributes, which may include
levels of priority for servicing, a specific arrival discipline or a particular
service rate. Incorporating these characteristics allows for more realistic
modelling and addresses the variety exhibited by agents in real-world queues.
Defining attributes such as these is straightforward in a computer simulation,
but this scenario is certainly not restricted to empirical studies. For
instance, in~\cite{RomeroSilva2017}, the authors provide empirical,
simulation-based evidence to dispel the idea that customers in multi-class
queues experience identical waiting time distributions. Meanwhile, the authors
of~\cite{Pompigna2020} demonstrate how a purely theoretical model of a motorway
tollgate is improved by the use of multiple customer classes, and is verified
using simulation. Finally, in~\cite{Reveil2014}, the authors derive a model that
groups arrivals from the same class together by way of a two-stage arrival
process. The motivation for grouping arrivals is to provide a schema which
supports the common assumption that reducing the number of times a server must
change the task they are completing will influence the overall throughput of
that server; the article succeeds in demonstrating this.

\subsubsection{Queuing and clustering}

The beginning of Section~\ref{subsubsec:queuing} briefly mentions two methods
for investigating patient pathways --- process mining and patient flow analysis
--- that are underpinned by a queuing network. In each case, the healthcare
system is represented as a network of nodes through which patients and resources
move at certain rates, as in a queuing network. Patient flow analysis considers
the entire network, while process mining attempts to identify critical parts of
the network. 

The modelling of patient pathways is of great importance to understanding
patient care in a healthcare system. Recording and studying how a patient moves
through the system may provide insights into the dynamics of the characteristics
presented by individuals, and how their true pathway differs from any previously
defined clinical ideal. Two recent surveys on the literature surrounding the
modelling of clinical pathways are~\cite{Aspland2019,DeRamonFernandez2019} with
the latter focusing on process management tools such as process mining.

Often, clustering forms a fundamental part of these methodologies, creating a
significant overlap. Clustering is a common resort in scenarios such as
this where something must be implemented to rationalise an otherwise vast,
heterogeneous set of behaviours. As an example, the authors
of~\cite{Prokofyeva2020} use hierarchical clustering to group clinical pathways
which, in turn, informs a broader, adaptive patient flow model. Meanwhile, the
authors of~\cite{Rebuge2012} consider a pathway as a sequence of events sampled
from a Markov chain. These chains (and their respective transition matrices)
define the clusters in the sequence data, in accordance with the methodology
presented in~\cite{Cadez2000}.

Other than process mining, clustering has acted as a catalyst for optimising
other healthcare processes which relate to the efficient treatment of patients.
Namely, the scheduling and planning of resources, as exemplified
in~\cite{Nasir2018,Steins2013,Yousefi2020}. In each of these works, some form of
clustering is used to categorise patients according to their requirements. This
clustering then informs the scheduling model, providing a richer and more
sophisticated picture of the overall system. With the other methods discussed in
this survey, clustering exclusively sits in the post hoc stage of analysis. In
process mining, a generalisation is formed for patient pathways. In segmentation
analysis, clustering provides a patient profiling scheme.

Recalling the benefits of multi-class queues --- being able to better model the
variety of agents in real-world queues --- it appears that clustering would be a
natural fit to inform a model. Segmentation analysis would be able to quantise
the idiosyncrasies of patients and quantify their effect on attributes of
queuing networks, including arrival behaviour, transition probabilities,
priority levels, and service requirements. However, the literature survey
revealed scarce research into the application of clustering to inform a
multi-class queuing model, theoretical or otherwise.


\subsection{Evaluating a model}

\begin{itemize}
    \item General overview of evaluation
    \item Objectivity vs. fairness
    \item Human intervention
    \item Unconscious bias
\end{itemize}

\subsubsection{Clustering}

Often lumped in with the task of
classification, clustering is distinct in that it reveals the underlying
structure of a dataset with knowledge that is fixed from the outset exclusively.
The alternative being a method with access to some ground truth to refer to,
such as a labelling.

\begin{itemize}
    \item internal or external metrics
    \item mean-squared error (inertia), silhouette, other label-invariant
        metrics
    \item information retrieval metrics (fundamentally flawed; see above)
    \item deriving agnostic cluster quality measures
    \item fair clustering methods
    \item benchmark datasets (often classification datasets)
    \item synthetic datasets (benchmarks(?), how are they made?)
\end{itemize}

\subsubsection{Healthcare settings}

\begin{itemize}
    \item often dependent on model
    \item incorporating staff and patient feedback
    \item where is model accountability?
\end{itemize}
